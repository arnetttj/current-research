% !TEX TS-program = pdflatex
% !TEX encoding = UTF-8 Unicode

% This is a simple template for a LaTeX document using the "article" class.
% See "book", "report", "letter" for other types of document.

\documentclass[11pt]{article} % use larger type; default would be 10pt

\usepackage[utf8]{inputenc} % set input encoding (not needed with XeLaTeX)

%%% Examples of Article customizations
% These packages are optional, depending whether you want the features they provide.
% See the LaTeX Companion or other references for full information.

%%% PAGE DIMENSIONS
\usepackage{geometry} % to change the page dimensions
\geometry{a4paper} % or letterpaper (US) or a5paper or....
% \geometry{margin=2in} % for example, change the margins to 2 inches all round
% \geometry{landscape} % set up the page for landscape
%   read geometry.pdf for detailed page layout information

\usepackage{graphicx} % support the \includegraphics command and options

% \usepackage[parfill]{parskip} % Activate to begin paragraphs with an empty line rather than an indent

%%% PACKAGES
\usepackage{booktabs} % for much better looking tables
\usepackage{array} % for better arrays (eg matrices) in maths
\usepackage{paralist} % very flexible & customisable lists (eg. enumerate/itemize, etc.)
\usepackage{verbatim} % adds environment for commenting out blocks of text & for better verbatim
\usepackage{subfig} % make it possible to include more than one captioned figure/table in a single float
\usepackage{amssymb}
\usepackage{amsmath}
\usepackage{algorithmicx}
\usepackage{algorithm}
\usepackage{graphicx}
\usepackage[noend]{algpseudocode}
\usepackage[hidelinks]{hyperref}
\let\oldemptyset\emptyset
\let\emptyset\varnothing

\newcommand{\vDashv}{%
  \mathrel{%
    \text{%
      \ooalign{$\vDash$\cr\reflectbox{$\vDash$}\cr}%
    }%
  }%
}
\makeatletter
\def\BState{\State\hskip-\ALG@thistlm}
\makeatother
% These packages are all incorporated in the memoir class to one degree or another...

%%% HEADERS & FOOTERS
\usepackage{fancyhdr} % This should be set AFTER setting up the page geometry
\pagestyle{fancy} % options: empty , plain , fancy
\renewcommand{\headrulewidth}{0pt} % customise the layout...
\lhead{}\chead{}\rhead{}
\lfoot{}\cfoot{\thepage}\rfoot{}

%%% SECTION TITLE APPEARANCE
\usepackage{sectsty}
\allsectionsfont{\sffamily\mdseries\upshape} % (See the fntguide.pdf for font help)
% (This matches ConTeXt defaults)

%%% ToC (table of contents) APPEARANCE
\usepackage[nottoc,notlof,notlot]{tocbibind} % Put the bibliography in the ToC
\usepackage[titles,subfigure]{tocloft} % Alter the style of the Table of Contents
\renewcommand{\cftsecfont}{\rmfamily\mdseries\upshape}
\renewcommand{\cftsecpagefont}{\rmfamily\mdseries\upshape} % No bold!

\newcommand{\Dashv}{\reflectbox{$\models$}}
%%% END Article customizations

%%% The "real" document content comes below...

\title{Current Research}
\author{Tim Arnett}
\date{Summer 2017} % Activate to display a given date or no date (if empty),
         % otherwise the current date is printed 

\begin{document}
\maketitle

\tableofcontents

\section{Hierarchical Fuzzy Inference Systems (hFIS) vs. Single Fuzzy Inference System (sFIS)}
\begin{itemize}
\item \textbf{Efficacy of hFIS compared to sFIS}

In order to show the differences between an hFIS and sFIS, a method was developed during the summer of 2016 to convert an hFIS to an sFIS (ref). This method was based on an analysis of the current active mode within the FISs and equating the output membership functions. The membership functions of the sFIS could then be described as a function of the output membership functions of the hFIS. The final representation of the sFIS equivalent to an hFIS is shown in Eq. (\ref{cascade2fullgeneral})-(\ref{Ku2}).

\begin{align}\label{cascade2fullgeneral}
\begin{split}
y = \mu_{k}&\left[ \begin{array}{cc}
\mu_{j} & \mu_{j+1}
\end{array} \right] 
K_{u_{1}} RB_1
\left[ \begin{array}{cc}
\mu_{i} \\
\mu_{i+1}
\end{array} \right]+\\
\mu_{k+1}&\left[ \begin{array}{cc}
\mu_{j} & \mu_{j+1}
\end{array} \right] 
K_{u_{2}} RB_2 
\left[ \begin{array}{cc}
\mu_{i} \\
\mu_{i+1}
\end{array} \right]
\end{split}
\end{align}

\begin{align}\label{RB1general}
\begin{split}
& RB_1 = \\
& \left[\begin{array}{r}
\left(U_{l+1,j}-U_{l,j}\right) \left[\begin{array}{cc} 
U_{i} & U_{i+1} 
\end{array}\right] \\
\left(U_{l+1,j+1}-U_{l,j+1}\right)\left[\begin{array}{cc} 
U_{i} & U_{i+1} 
\end{array}\right] 
\end{array}\right] + \\
&\left[\begin{array}{r} \left(y_{1_{l+1}}U_{l,j}-y_{1_{l}}U_{l+1,j}\right)\left[\begin{array}{cc} 1 & 1 \end{array}\right] \\ 
\left(y_{1_{l+1}}U_{l,j+1} y_{1_{l}}U_{l+1,j+1}\right)\left[\begin{array}{cc} 1 & 1 \end{array}\right] \end{array}\right]+ \\
& \left(y_{1_{l+1}}-y_{1_{l}}\right)\left(y_{2_{m+1}}U_{m,k}-y_{2_{m}}U_{m+1,k}\right)
\left[ \begin{array}{cc}
1 & 1 \\
1 & 1 \end{array}\right]
\end{split}
\end{align}

\begin{align}\label{RB2general}
\begin{split}
& RB_2 = \\
 & \left[\begin{array}{r}
\left(U_{l+1,j}-U_{l,j}\right)\left[\begin{array}{cc}
U_{i} & U_{i+1} \end{array}\right] \\
\left(U_{l+1,j+1}-U_{l,j+1}\right)\left[\begin{array}{cc}
U_{i} & U_{i+1} \end{array}\right]
\end{array}\right]+\\
& \left[\begin{array}{r}\left(y_{1_{l+1}}U_{l,j}-y_{1_{l}}U_{l+1,j}\right)\left[\begin{array}{cc} 1 & 1 \end{array}\right] \\ 
\left(y_{1_{l+1}}U_{l,j+1}-y_{1_{l}}U_{l+1,j+1}\right)\left[\begin{array}{cc} 1 & 1 \end{array}\right]
\end{array}\right]+ \\
& \left(y_{1_{l+1}}-y_{1_{l}}\right)\left(y_{2_{m+1}}U_{m,k+1}-y_{2_{m}}U_{m+1,k+1}\right)
\left[ \begin{array}{cc}
1 & 1 \\
1 & 1 \end{array}\right]
\end{split}
\end{align}

\begin{equation}\label{Ku1}
K_{u_{1}} = \left(\frac{1}{y_{1_{l+1}}-y_{1_{l}}}\right)\left(\frac{U_{m+1,k}-U_{m,k}}{y_{2_{m+1}}-y_{2_{m}}}\right)
\end{equation}

\begin{equation}\label{Ku2}
K_{u_{2}} = \left(\frac{1}{y_{1_{l+1}}-y_{1_{l}}}\right)\left(\frac{U_{m+1,k+1}-U_{m,k+1}}{y_{2_{m+1}}-y_{2_{m}}}\right)
\end{equation}

As complex as this appears, a closed form for a more general hFIS to sFIS representation may be possible. This transformation may be useful after dynamic hFIS structure learning to then convert back to an sFIS to more closely approximate ideal control mappings. However, this transformation does not explore the limitations of the approximation abilities of the hFIS as compared to the sFIS. The constraints on the FISs used during this work were found to be called Ruspini Partitioning (ref ruspini 1969). A paper showing universal approximation abilities of an sFIS with these properties was found (ref) but did not explore the 

\item 
\item 
\end{itemize}
\section{Dynamic hFIS structure}

\section{Weekly Record}
\subsection{5/15/17-5/19/17}

\section{Stream of Consciousness}
\subsection{Learning underlying target function structure during reinforcement learning}
\begin{itemize}
\item Current criteria for identifying when an sFIS should attempt to split into a particular hFIS are suboptimal (understatement). Currently utilizing the coefficients of the polynomials for each mode of the current best set of FIS parameters (see (ref)).
\item What other methods can be used to quickly identify portions (or all) of the underlying optimal control function? This is starting to blur lines between Machine Learning, optimal control, and system/parameter ID.
\item Can Neural Nets be used in conjunction with FISs (maybe as the activation functions) to produce good results? Is that useful? Can Convolutional Neural Nets be used for feature extraction in the solution space during reinforcement learning? If so can that be used to adjust the structure of the hFIS in a fast and meaningful way during reinforcement learning?
\item There's some work on learning Bayesian network structure. Can we use those methods or perhaps Bayesian methods themselves to refine hypotheses (solutions of desirable parameter sets) on the fly during reinforcement learning? Possibly major issues involving establishing probabilities of better reward/fitness for particular actions.
\item What other methods can be used in conjunction with reinforcement learning? Principal component analysis/projection pursuit? Can Artificial Immune Systems be used since they use rules that drive the learning process (FIS-like)?
\end{itemize}


\end{document}
